\section{Related Work} \label{sec:related-work}

% Simulating using multiple models is a novelty in computer systems; however, as introduced in \Cref{sec:introduction}, other scientific fields already employ multi-model simulations, such as virology~\cite{covid19ForecastHubUS}, weather and climate simulation~\cite{schunk2016space-multimodel-weather}, or ecology~\cite{ecologyMultiModel}. In ICT, existing simulators currently rely on single model predictions~\cite{DBLP:conf/ccgrid/MastenbroekAJLB21, DBLP:journals/spe/CalheirosRBRB11, DBLP:journals/spe/HewageIRB24, DBLP:conf/ccgrid/Casanova01, DBLP:journals/fgcs/McDonaldDWSC24}. We are the first to propose Multi- and Meta- Model simulation of ICT infrastructure.

% Closest to our work, we identify COVID-19 Forecast Hub~\cite{covid19ForecastHubUS}, used to predict the COVID-19 hospitalizations in the United States, and providing 93 models, out of which users can select and leverage in a unified visualization. This is equivalent to the \textit{Multi-Model} component of M3SA. In weather and climate simulation~\cite{environmentMultiModel}, Myhre et al. investigated the geographical distribution of harmful emissions using seven models unified under a single representation and aggregated (averaged) into a ``Mean Model'', helping to contrast biases and improve prediction explainability. This is equivalent to the \textit{Meta-Model} component of M3SA.

% Datacenter simulators are useful tools for the community and are much used in predicting ICT infrastructure under workload. However, no simulator provides multi- and meta-simulation capabilities but relies on single-model predictions~\cite{DBLP:conf/ccgrid/MastenbroekAJLB21, DBLP:journals/spe/HewageIRB24, DBLP:journals/spe/CalheirosRBRB11, DBLP:conf/ccgrid/Casanova01, DBLP:journals/fgcs/McDonaldDWSC24}. Our work adds this capability as an external add-on.


Image classification using deep learning has been extensively explored in recent years with many different ways to make models more efficient \cite{obaid2020deep} and provide state-of-the-art results. A widely accepted benchmark for evaluating these improvements is the ImageNet dataset, as demonstrated in the work of Mishkin et al. \cite{MISHKIN201711}, where different architectures are systematically analyzed to assess their effectiveness. In our work, we also utilize ImageNet-1K as the benchmark dataset to evaluate the trade-off between accuracy and efficiency in deep learning.

Significant progress has been made in the field of machine learning to develop models that achieve state-of-the-art accuracy. For instance, ResNet \cite{He_2016_CVPR} which uses skip connections, allowing deep networks to mitigate the vanishing gradient problem and enabling the training of significantly deeper architectures.

Besides, achieving computational efficiency has been a key focus in recent years as well. Models such as MobileNet~\cite{DBLP:journals/corr/HowardZCKWWAA17} introduce depthwise separable convolutions to significantly reduce the number of parameters and computational cost while maintaining competitive accuracy.

However, while significant advancements have been made in either improving accuracy or enhancing efficiency, the trade-off between these two aspects has not been thoroughly explored. In our work we evaluate this trade-off between computational resources and accuracy, as well as the inference-accuracy trade-off, aiming to provide insights into the balance between architectural complexity and model performance.



% % \todosacheen {You just repeat your claims from the first paragraph in the last paragraph of the related work.}

% % Mai trebuie spus ceva, in primul paragraf, despre ce fac celelalte simulatoare de care stim + ca noi suntem primii care fac asta in domeniul asta. Scurt si fara alte adjective.
% % E bine cu alternativele din alte domeniu, dar prea detailiat. Si nu spune mid-scale sciences, ca oamenii nu (prea) stiu ce e aia. Vezi cum am formulat deja mai devreme in articol si spune "as introduced in §x, ...".


% % Totul scurt: 1§ overview -- suntem primii. §2 closest to our work from other sciences; toate acolo, pe scurt. §3 despre alte simulatoare -- stim care sunt, stim ca sunt bune "useful tools for the community, much used. However, they do not currently provide multi- and meta-simulation capabilities. Our work adds this capability as an external add-on."

% % In this section, we describe two examples of how other sciences successfully adopt simulation concepts proposed by M3SA.

% % In virology, the COVID-19 Forecast Hub, supported by American national health institutions, predicts COVID-19 hospitalizations in the United States. Although starting with only 10 models in April 2020, the tool proved its effectiveness and rapidly grew to 93 models by May 2022. Users can select models, and the tool overlays their predictions in a single visualization, helping to contrast biases and improve prediction explainability. This is equivalent to the \textit{Multi-Model} component of M3SA.

% % In weather and climate simulation, Myhre et al. investigate the geographical distribution of emissions (e.g., nitrogen oxides, black carbon)~\cite{environmentMultiModel}, using seven models from the EU project ECLIPSE fed with the same datasets and unified under a single representation. To obtain a robust central estimate and balance out individual model biases, the authors developed a new model called ``Model Mean'' which averages predictions of other models. This is equivalent to the \textit{Meta-Model} component of M3SA, capable of mean aggregation of predictions.


% % Simulating using multiple models is a novelty in computer
% % systems; conducting multi-model simulations for datacenters
% % is also innovative beyond computer systems science. Coarse-
% % grained, large-scale models in weather and climate simula-
% % tion [20], [28] and detailed, small-scale models in ecology [29]
% % already use simulation instruments that rely on multiple
% % prediction models, calibrated and adjusted for the needs of
% % their scientific field. In contrast, to obtain meaningful pre-
% % dictions about complex datacenters ecosystems, medium-scale
% % models must combine higher-level abstraction with detailed
% % operational models for specific devices and applications, and
% % address various operational phenomena. This is fundamentally
% % different than other-scale sciences [30].
