

\begin{abstract}
  Artificial Intelligence is vital for the increasingly digitalized society, but represents a considerable fraction of global energy consumption and involves massive financial resources. To improve the environmental and economic sustainability of AI workloads, researchers and engineers sacrifice accuracy, choosing simpler architectural models with lower accuracy, for non-life-threatening operations. It is critical yet non-trivial to address the trade-off between architectural complexity and the performance of AI models.
  To address this challenge, in this work we design, implement, and compare SiMo and CoMo, two models of contrasting architectural complexities. SiMo is a \underline{Si}mple architectural \underline{Mo}del, lightweight to train and host, encapsulating only a shallow, yet functional CNN; CoMo is a \underline{Co}mplex architectural \underline{Mo}del, leveraging 4 pretrained models and aggregating their predictions. We select object classification as the application domain.
  Through experiments with prototypes we i) evaluate the trade-off between training resources and accuracy, ii) evaluate the inference-accuracy trade-off, and iii) compare the accuracy of individual models aggregated by CoMo with the accuracy of CoMo itself, the composite model. This research is provided as open science and is available on the Weights and Biases interactive platform.
\end{abstract}

\begin{IEEEkeywords}
machine learning, architectural complexity, complexity-accuracy tradeoff, complexity-performance tradeoff, SiMo, CoMo
\end{IEEEkeywords}






% \begin{abstract}
% Datacenters are vital for the digital society and represent a considerable fraction of global energy consumption, estimated at 3\%, and are expected to rise to 8\% by 2030, further reducing already over-exploited resources. In the operations of every datacenter, simulators play a crucial role in predicting the capabilities of real or virtual infrastructure, under various workloads. Although many simulators proved useful in lowering energy consumption, the current state-of-the-art is based on singular models embedded in either simulation or analytical frameworks, that offer good predictive capabilities only for the limited context in which they were developed. We propose M3SA, a simulation analysis tool able to leverage multiple models and combine their results. M3SA filters out extremes and computes the most likely curve or range of values over time. By employing multi-model processes, we can develop more efficient simulators, which would aid in developing, configuring, and operating efficient, energy-conserving, and cost-friendly datacenters.
% \end{abstract}
