

\begin{abstract}
Datacenters represent the backbone of our digital society, but raise operational challenges when workload grows.
We envision digital twins as becoming primary instruments in datacenter operation, continuously and autonomously adapting ICT infrastructure on demand with a human-in-the-loop for major decisions.
However, unlike other fields that successfully employ digital twins (e.g., aviation, autonomous driving), and albeit its critical importance in ICT operation, a digital twinning ecosystem for operating and monitoring ICT infrastructure has never been proposed.
Addressing this challenge, we propose \underline{OpenDT}, the first \underline{Open}-source, \underline{D}igital \underline{T}winning Ecoystem for monitoring and operating datacenters through a continuous integration cycle:
(1) live and continuous telemetry data;
(2) discrete-event simulation using a digital twin of the physical ICT infrastructure, adapted to the live telemetry data, and evaluated against Service Level Objectives (SLOs);
(3) LLM-based, and human-approved feedback to physical ICT; and
(4) adjustment of the phyiscal ICT following the feedback.
In this work, our contribution is twofold: firstly, we propose a high-level design of OpenDT against established functional and non-functional requirements; secondly, adhering to state-of-the-art practices, we engineer a prototype of OpenDT.
\end{abstract}

% The framework for digital twinning ICT ecosystems in the Nethelands has been set by AtLarge Team ... A reference architecture has been proposed and developed by the AtLarge Research Team and partners in the Groeifonds programma 6G FNS.

% This project is the first to set up ...

% 1) Unseen OpenDC Scenarios
% 2) LLMs in scenario/topology, and data extraction. LLMs that make the input info structured and sets up the simulator
% 3) Multi-Model simulation in the simulation component

\begin{IEEEkeywords}
datacenters, digital twins, simulation, performance, energy utilization, OpenDT
\end{IEEEkeywords}


 % Datacenters represent the backbone of our digital society, but raise operational challenges when workload grows. 
  % We envision digital twins as becoming primary decision-making instruments for datacenter operation, continuously adapting infrastructure on demand following a human-in-the-loop paradigm.
  % However, unlike other fields that successfully employ digital twins (e.g., aviation, autonomous car driving), it has never been proposed a digital twinning system of ICT infrastructure under workload. 
  % Addressing this challenge, we propose \underline{OpenDT}, the first \underline{Open}-source, end-to-end, workload-aware \underline{D}igital \underline{T}winning System for monitoring and operating datacenters designed with a continous integration cycle: 
  % (1) live and continuous telemetry data, 
  % (2) discrete-event simulation using as input a digital twin of the physical ICT infrastructure, telemetry data, and defined Service Level Objectives (SLOs), 
  % (3) feedback to the physical ICT, and 
  % (4) adjustment of the physical infrastructure based on the system's feedback.



