\IEEEoverridecommandlockouts
% The preceding line is only needed to identify funding in the first footnote. If that is unneeded, please comment it out.
\usepackage{cite}
\usepackage{amsmath,amssymb,amsfonts}
\usepackage{booktabs}
\usepackage{algorithmic}
\usepackage{graphicx}
\usepackage{adjustbox}
\usepackage[inline]{enumitem}
\usepackage{textcomp}
\usepackage{lipsum}
\usepackage{xcolor}
\usepackage{tikz}
\usepackage{float}
\usepackage{placeins}
\usepackage{listings}
\usepackage{pifont}
\PassOptionsToPackage{hyphens}{url}\usepackage{hyperref}
\usepackage{cleveref}
\setlength{\marginparwidth}{2cm}
\usepackage[colorinlistoftodos,prependcaption,textsize=tiny]{todonotes} % \usepackage[colorinlistoftodos,prependcaption,textsize=tiny]{todonotes} to disable the notes
\usepackage{xargs}                    
\newcommand{\todocite}[1]{\todo[color=green!40]{#1}}
\usepackage{tabularray}
\usepackage{pifont}
\usepackage{tabularx}

\newcommand\blfootnote[1]{%
  \begingroup
  \renewcommand\thefootnote{}\footnote{#1}%
  \addtocounter{footnote}{-1}%
  \endgroup
}

\newcommandx{\todolarge}[2][1=]{\todo[inline,size=\large,linecolor=blue,backgroundcolor=cyan,bordercolor=blue,#1]{#2}}
\newcommand{\todoother}[1]{\todo[inline, color=cyan!40]{Other: #1}}
\newcommand{\todoall}[1]{\todo[inline, color=orange!40]{TODO: #1}}
\newcommand{\todoradu}[1]{\todo[inline,color=teal!40]{Radu: #1}}
\newcommand{\todoalexis}[1]{\todo[inline, color=gray!40]{Alexis: #1}}
\newcommand{\todomahesh}[1]{\todo[inline, color=green!40]{Mahesh: #1}}
\newcommand{\todoben}[1]{\todo[inline, color=purple!40]{Ben: #1}}



% Citation own system
\newcommand{\citationstodo}[1]{\textsuperscript{\color{blue} [#1]}}
% Citation needed
\newcommand{\citationneeded}{\textsuperscript{\color{blue} [citation needed]}}
\newcommand{\citationsneeded}[1]{\textsuperscript{\color{blue} [#1 needed]}}


%%% \verify, \cut, \addref, \unsure, \change, \info, \improvement, \thiswillnotshow
\newcommand{\verify}[1]{\todo[inline, color=cyan!40]{check: #1}}
\newcommand{\cut}[1]{\todo[inline, color=yellow!40,disable]{Cut: #1}}
\newcommandx{\addref}[2][1=]
{\todo[inline,linecolor=blue,backgroundcolor=blue!50,bordercolor=blue,#1]{Add reference. #2}}
\newcommandx{\unsure}[2][1=]{\todo[inline, linecolor=red,backgroundcolor=red!25,bordercolor=red,#1]{#2}}
\newcommandx{\change}[2][1=]{\todo[inline, linecolor=blue,backgroundcolor=blue!25,bordercolor=blue,#1]{#2}}
\newcommandx{\info}[2][1=]{\todo[linecolor=OwnOliveGreen,backgroundcolor=OwnOliveGreen!25,bordercolor=OwnOliveGreen,#1]{#2}}
\newcommandx{\improvement}[2][1=]{\todo[linecolor=Plum,backgroundcolor=Plum!25,bordercolor=Plum,#1]{#2}}
\newcommandx{\thiswillnotshow}[2][1=]{\todo[disable,#1]{#2}}
%

\definecolor{darkred}{rgb}{0.5,0,0}
\definecolor{darkgreen}{rgb}{0,0.5,0}
\definecolor{darkblue}{rgb}{0,0,0.5}
\newcommand{\note}[1]{\noindent\textbf{\textsc{\textcolor{darkgreen} {(NOTE: #1)}}}}
\newcommand{\NOTE}[1]{\noindent\textbf{\textsc{\textcolor{red} {(IMPORTANT: #1)}}}}

\newcommand{\circled}[1]{%
    \tikz[baseline=(char.base)]{
        \node[shape=circle, fill=black, text=white, inner sep=0.5pt, font=\small] (char) {#1};
    }%
}

\def\BibTeX{{\rm B\kern-.05em{\sc i\kern-.025em b}\kern-.08em
    T\kern-.1667em\lower.7ex\hbox{E}\kern-.125emX}}

\makeatletter
\setlength{\@fptop}{0pt}
\makeatother